% vi: ft=plaintex sw=2 sts=2

\chapter{Introduction}

\section{WebSocket}

WebSocket is a new web technology for full-duplex communication between web
browsers and web servers\fnurl{}
{https://developer.mozilla.org/en-US/docs/WebSockets}. Normally a web
browser fetches data via HTTP GET requests and transmits data to the server via
a HTTP POST request, which then can also contain response data. The downside is
that each request and response causes \fnurl{considerable overhead}
{http://blog.kaazing.com/2010/02/24/5-signs-you-need-html5-web-sockets-part-2/}, 
since the HTTP protocol has been designed for transmitting web pages and not
real-time data.  There also is no mechanism for a server to push data to the
client without the client initiating an HTTP request.

There are workarounds for these problems, like long-lived GET and POST requests,
but these have their own downsides.

WebSocket is comparable to TCP sockets in that it offers a standing connection
that allows sending and receiving information without a request-response cycle.

\subsection{Opportunities and problems}

The WebSocket standard is part of a larger trend of the web client becoming more
powerful. One possible application of WebSocket in combination with the
\fnurl{\emph{DOM Storage API}}
{https://developer.mozilla.org/en-US/docs/DOM/Storage} is the use of client-side
data management where the server can push data to the client and the client can
send small messages to the server with little overhead.

A common scenario is a client reading some entries from the server, performing a
possibly time-consuming task with the read data, and storing a result on the
server. The question then arises: what should happen when the input data changed
in the meantime?

One way to deal with this is by locking the input values and not allowing
other clients to change them until the task is completed.

This has several downsides: keeping locks is expensive and not always necessary.
Additionally, there is no way for another client to know if a write lock is
actually used or only accidentally granted.

\section{Optimistic concurrency with Kung and Robinson}

An alternative to locking is the use of transactions as described in Kung and
Robinsons paper \emph{On optimistic methods for concurrency control}
\cite{kung1981}.

The paper presents a simple protocol for applying transactions: each transaction
has a \emph{read set} of entries that are expected to not have changed, a set of
entries to be written in case the input values haven't changed, and a
transaction number \emph{startTn} that indicates when the reads have been made.

When applying the transaction, the server checks if there have been any changes
to the \emph{read set} since \emph{startTn} by checking the writes of the
transactions between \emph{startTn} and the current transaction number (the
number of the least recently applied transaction). If this condition holds true,
the changes declared in the \emph{write set} are applied, the transaction number
is increased and the transaction is stored internally in order to validate new
transactions.

Kung and Robinson also defined the operations \emph{create} and \emph{delete},
which can be omitted in this project because low-level memory management is not
a priority on modern systems.

\subsection{Offline transactions}

In \emph{Mobile Computing} by Prof. Dr. Th. Fuchß \cite{fuchss2009}, a method of
maintaining a consistent database is described where the client is disconnected
from the database for a period of time while still generating transactions.
Instead of applying the transaction immediately, the client stores the
transaction locally and requests a reintegration of all locally isolated
transactions on reconnect.

\pagebreak
\subsection{Examples}

\subsubsection{Applications of single transactions}

\begin{table}[h]
\centering
\begin{tabular}{|lllll|}
  \hline
  startTn & Reads & Writes & Valid & assigned Tn\\ \hline
  0       & a     & b: 1, c: 2  & true  & 1 \\
  0       &       & b: 2        & true  & 2 \\
  0       & a, b  & b: 3        & false & - \\
  2       & a, b  & b: 3, a: 1  & true  & 3 \\
  \hline
\end{tabular}
\caption{application of single transactions}
\label{tab:application-of-single-transactions}
\end{table}

The third transaction failed because \emph{b} has been modified since
\emph{startId} 0. The final data table on the server is \emph{a: 1, b: 3, c: 2}.

\subsubsection{Reintegration of a collection of transactions}

The next tables describes a scenario where offline isolated transaction are
reintegrated into the remote database. Note that the transaction number
(\emph{\#}) is only used internally and is not relevant for the reintegration,
where a global \emph{startId} is passed alongside the collection of
transactions.

\begin{table}[h]
\centering
\begin{tabular}{|lll|}
  \hline
  \# & Reads & Writes \\ \hline
  1 & a     & b: 1 \\
  2 &       & c: 2 \\
  3 & a, b  & b: 3 \\
  4 & c     & b: 4 \\
  \hline
\end{tabular}
\caption{isolated transactions}
\label{tab:isolated-transactions}
\end{table}

The assumed transactions on the remote database

\begin{table}[h]
\centering
\begin{tabular}{|ll|}
  \hline
  Tn & Writes \\ \hline
  4  & a: 1, b: 1 \\
  5  & a: 2 \\
  6  & b: 5 \\
  \hline
\end{tabular}
\caption{remote writes}
\label{tab:remote-writes}
\end{table}

The reintegration of these transactions starting at \emph{startTn = 4} checks
modifications by other modifications beginning at the transaction with the id
\emph{5} (to stay consistent with the use of \emph{startTn} in the single
transaction application mode). Only transactions \#2 and \#4 succeed. The local
transactions \#1 and \#3 fail due to the remote transaction \#5.  The resulting database entries are \emph{a: 1, b: 4, c: 2}
