% vi: ft=plaintex sw=2 sts=2

\chapter{Technologies}

The following sections will describe the technologies needed to implement a
simple web application that allows creation and submission of transactions and
proper handling of isolated transactions that are submitted while being
disconnected from the server. The database is a simple key-value store on the
server side, the client will consist of a simple form to add entries to the
\emph{read} set and add pairs to the \emph{write} set.

The client should also display a table of the locally isolated transactions and
the transaction application attempts on the remote database. This isn't strictly
necessary for production use but helpful in a demonstration project.

The validation and application of transactions is done centrally on a server
that accepts \emph{WebSocket} requests. Clients must call the remote methods
|applyTransaction| and |reintegrateTransactions| to apply their transactions.

\section{CoffeeScript}

\fnurl{CoffeeScript}{http://www.coffeescript.org} is a language that compiles to
JavaScript. The JavaScript language is widely considered as lacking and
partially flawed, which has spawned many other languages that aim to provide
languages with better semantics that then compile to JavaScript. Most of the
syntax is easily understandable for anyone who has programmed in C, Java or
JavaScript, but there are some aspects that should be explained in detail:

\begin{description}
\item[Implicit local scope]
In JavaScript, values are implicitly global if not declared with the
|var| statment. CoffeeScript has the inverse approach and
treats values as implicitly function-scoped unless explicitly declared as
global by attaching them to the global object (e.g. |window| in the browser).
This means that the |var| keyword can be dropped.


\item[Shorthand for \emph{this}]
The |this| operator in JavaScript returns a reference to the current function.
CoffeeScript provides a shorthand by allowing to write |@attr| instead of
|this.attr|.

For more information on the JavaScript |this| operator, refer to the
\fnurl{Mozilla Developer Network}
{https://developer.mozilla.org/en-US/docs/JavaScript/Reference/Operators/this}

\item[Improved function declaration notation]
The way to declare a function in JavaScript is to type out the keyword
|function|, which gets tiresome quickly. Example in JavaScript

\begin{minted}{coffeescript}
var add = function (a, b) { return a + b }
\end{minted}

CoffeeScript offers the shorthand |->|

\begin{minted}{coffeescript}
add = (a, b) -> return a + b
\end{minted}

The "fat-arrow" notation |=>| preserves the scope of |this|

\begin{minted}{coffeescript}
func = ->
    @attr = 1
    x = => @attr
    x() # is 1
\end{minted}

\item[Implicit return]
Another improvement is considering the last return value of a statement in a
function as the return value of the function. To improve our previous example,
it is now possible to write

\begin{minted}{coffeescript}
add = (a, b) -> a + b
\end{minted}

\item[Significant whitespace]
As indicated in the last example, C-style curly braces are also optional and can
be replaced with significant whitespace. Code blocks are then indicated with
indentation:

\begin{minted}{coffeescript}
foo = ->
    # some code
    bar = ->
        # some other code
        # part of the "bar" scope

    # this is part of the "foo" scope
\end{minted}

\item[Object notation]
Another instance of optional curly braces is an improvement on the object
notation: JavaScript Objects are hash tables that can be declared by the
notation

\begin{minted}{javascript}
var o = {a: 1, b: 2}
\end{minted}

CoffeeScript also accepts the forms

\begin{minted}{coffeescript}
o = a: 1, b: 2
\end{minted}

and

\begin{minted}{coffeescript}
o =
    a: 1
    b: 2
\end{minted}


\item[Nearly everything is an expression]
Many constructs that are traditionally regarded as statements in other languages
can be used as expressions in CoffeeScript. For example

\begin{minted}{coffeescript}
max = (a, b) -> if (a > b) then a else b
\end{minted}

\item[Implicit parentheses]
The last significant deviation from C-style syntax is the use of implicit
braces. Instead of writing

\begin{minted}{coffeescript}
result = max(a, b)
\end{minted}

you can use the lighter notation of

\begin{minted}{coffeescript}
result = max a, b
\end{minted}

\end{description}

\section{Underscore.js}

JavaScript's first-class functions allow for a functional style of programming.
Unfortunately, common primitives for functional programming like \emph{map} and
\emph{each} are lacking or inconsistently implemented across browsers. The
JavaScript library \emph{underscore.js} aims at patching the deficiencies of
standard JavaScript and providing a toolkit for functional programming.

Underscore methods are accessible through the underscore identifier: \emph{\_}

Example:

\begin{minted}{coffeescript}
result = _.map [1,2,3,4], (x) -> x * 2

# result is [2, 4, 6, 8]
\end{minted}

\begin{minted}{coffeescript}
result = _.all [1,2,3,4], (x) -> x < 4

# result is false since (4 < 4) is false
\end{minted}

A commonly used pattern for creating an object from another object uses the
function |_.reduce|. From the documentation

\begin{quote}
 |_.reduce(list, iterator, memo, [context])|

 Boils down a list of values into a single value. Memo is the initial state of
 the reduction, and each successive step of it should be returned by iterator.
 The iterator is passed four arguments: the |memo|, then the |value| and |index|
 (or |key|) of the iteration, and finally a reference to the entire list.

 \begin{minted}{coffeescript}
  sum = _.reduce(
    [1, 2, 3], ((memo, num) -> memo + num), 0
  )
  # sum = 6
 \end{minted}
\end{quote}

To read all values form a database and return a dictionary with |key: value|
mappings, |_.reduce| can be used as follows:

\begin{minted}{coffeescript}
  result = _.reduce(
   readKeys,
   ((obj, key) ->
    obj[key] = database.read key
    obj),
   {}
  )
\end{minted}

We will use this library on the client side as well as on the server. The
functional programming style greatly improves the readability of the described
algorithms.

\section{Node.js}

The server stack for our project is based on
\fnurl{\emph{Node.js}}{http://www.nodejs.org}, a popular
stand-alone implementation of JavaScript based on Chrome's \fnurl{\emph{V8
JavasScript Engine}}{http://code.google.com/p/v8/} engine.

Since the client-side API of WebSocket can only be used in JavaScript, this
allows the server to be written in the same language as the client and reduces
mental context switching when writing web applications. Another feature is
sharing code and libraries on the client and the server, which improves code
reuse and readability.

JavaScript is designed to run as a single thread, concurrency is achieved by
using events. This reduces the amount of locking and thread management on the
server and the client and thereby contributes to ease of development and overall
code quality.

\section{Zappa.js}

The web application framework \fnurl{\emph{Zappa.js}}{http://www.zappajs.org} is
a CoffeeScript-friendly adaptation of
\fnurl{\emph{express.js}}{http://www.expressjs.com}, a web application framework
on top of \emph{node.js}. A simple web application written in \emph{Zappa.js}
can be written as:

\begin{minted}{coffeescript}
require('zappajs') ->
  @get '/': ->
    @render index: {layout: no}

  @view index: ->
    doctype 5
    html ->
      head -> title "Hello World!"
      h1 "Hello World!"
\end{minted}

\pagebreak
A useful feature of \emph{Zappa.js} is the seamless integration of client-side
code that can be written in CoffeScript

\begin{minted}{coffeescript}
require('zappajs') ->
  # (... see previous sample)

  # expose script via path "/client.js"
  @client '/client.js': ->
    # greet client with an alert message
    alert "Hello World!"
\end{minted}

\section{Socket.io}

The JavaScript library \fnurl{\emph{socket.io}}{http://www.socket.io} provides
an abstraction over the raw WebSocket API on the client as well as on the server
side. This allows developers to use the same primitives in both environments,
greatly improving the usability of WebSocket.

The basic \emph{socket.io} structure in JavaScript on the server looks like this:

\begin{minted}{javascript}
var io = require('socket.io').listen(80);

io.sockets.on('connection', function (socket) {
  socket.emit('news', { hello: 'world' });

  socket.on('my other event', function (data) {
    console.log(data);
  });
});
\end{minted}

The client can communicate with the server and respond to server-side events via
the construct

\begin{minted}{javascript}
var socket = io.connect('http://localhost');
socket.on('news', function (data) {
  console.log(data);
  socket.emit('my other event', { my: 'data' });
});
\end{minted}

The communication in \emph{socket.io} is event-oriented. Connecting to the server
triggers the |connection| event. The callback function defined in the example
uses `socket.emit` to send the |'news'| event to the client, which logs the
transmitted data (|{hello: 'world'}|) to the console and in turn triggers
 |'my other event'|.


\pagebreak
\subsection{Socket.io integration in Zappa.js}

Utilizing the tight integration of client-side JavaScript running on the browser
into the server-side JavaScript, Zappa.js offers a tighter integration layer on
top of \emph{socket.io}

\begin{minted}{coffeescript}
require 'zappajs', ->
  @on message_to_server: ->
    console.log @data
    @emit message_to_client: 'pong'

  @client '/client.js': ->
    @connect()
    @emit message_to_server: 'ping'
    @on message_to_client: -> console.log @data
\end{minted}

In this example, the client connects to the server and sends the message of the
type |message_to_server| containing the value |'ping'| using the method |@emit|
(property of the function |@client|).

The server has registered a handler for the message type |message_to_server|
using the |@on| method, which in turn emits the message |message_to_client|.

It is also possible to return a response for an emitted message using the |@ack|
method, if the client emit method receives a function as the last parameter

\begin{minted}{coffeescript}
@on message_to_server: ->
  @ack "pong"

@client '/client.js': ->
  @connect()
  @emit messaget_to_server: "ping", -> console.log @data
\end{minted}

Here the client will display \emph{pong} in the browser console.
